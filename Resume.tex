\documentclass[margin,line]{res}
\usepackage{hyperref}
\usepackage{url}
\usepackage{color}
\oddsidemargin -.5in
\evensidemargin -.5in
\textwidth=6.0in
\itemsep=0in
\parsep=0in
\topmargin=0in
\topskip=0in
 
\newenvironment{list1}{
  \begin{list}{\ding{113}}{%
      \setlength{\itemsep}{0in}
      \setlength{\parsep}{0in} \setlength{\parskip}{0in}
      \setlength{\topsep}{0in} \setlength{\partopsep}{0in}
      \setlength{\leftmargin}{0.17in}}}{\end{list}}
\newenvironment{list2}{
  \begin{list}{$\bullet$}{%
      \setlength{\itemsep}{0in}
      \setlength{\parsep}{0in} \setlength{\parskip}{0in}
      \setlength{\topsep}{0in} \setlength{\partopsep}{0in}
      \setlength{\leftmargin}{0.2in}}}{\end{list}}


    
\begin{document}

\name{\LARGE Somin Wadhwa} \hfill {\em \today}

\begin{resume}
\section{\sc Contact Information}

\vspace{.05in}
\begin{tabular}{@{}p{3.5in}p{3in}}
Undergraduate Student             & {Phone:}  (+91) 9312349897 \\
Block 1, Computer Science \& Engineering 
 & {E-mail:}  sominwadhwa@gmail.com\\
Maharaja Agrasen Institute of Technology.\\
Rohini, Delhi, India.  & {GitHub:} {\href{https://github.com/sominwadhwa}{\color{blue} sominwadhwa}}
\end{tabular}


\section{\sc Interests}

Machine Learning, Data Analysis

\section{\sc Education}
{\bf B.Tech in Computer Science \& Engineering} \hfill July 2014 -- present\\
Maharaja Agrasen Institute of Technology \hfill(Overall Percentile: {\bf78.8\%} after 4 semesters)\\
%\vspace*{-.1in}
Guru Gobind Singh Indraprastha University, Delhi, India

{\bf High School}: Bal Bharati Public School, Pitampura, Delhi\hfill March 2012 -- April 2014\\
All India Senior School Certificate Examination, CBSE \hfill(Percentile: {\bf 93.8\%})\\
{\bf Secondary School}: Bal Bharati Public School, Pitampura, Delhi\hfill March 2000 -- April 2012\\
CBSE \hfill(GPA: {\bf 8.8})

\section{\sc Experience}
{\bf Summer Training (MOOC on Coursera)}*\hfill June, 2016 - September, 2016\\ \href{https://www.coursera.org/learn/machine-learning/home/welcome}{Machine Learning by Stanford University}\\  
11 Weeks of training in Machine Learning(Supervised \& Unsupervised) on Octave.

\section{\sc Technical Skills}
{\bf Strongest Areas}: Supervised Learning, Data Structures, Dynamic Programming\\
{\bf Languages}: Python, C++\\
{\bf Tools \& Frameworks}: Matlab, \LaTeX, NumPy \& Pandas, MS Office Suite  \\
{\bf Database Tools}: Oracle, MySql, sqllite 

\section{\sc Relevant \\Courses}
Data Structures \& Algorithms, Databases, Machine Learning, Automata Theory, Probability, 
Differential \& Inferential Statistics, Software Engineering 
%%%%%%%%%%%%%%%%%%%

\section{\sc Research Work}

%%%%%%%%%%%%%%%%%%%%%%%%%%%%%%%%%%%%%%%%%%%%%%%%%%%%%%%%%%%%%%%%%%%%%%%%%%%%%%%%%%%%%%%%%%%%%%%%%%%%%%%%%%%%%%%
{\bf Somin Wadhwa}, "Study of Random Numbers \& their applications in computational physics using Monte-Carlo method",{\em  XXVII IUPAP Conference on Computational Physics, IIT Guwahati}, December 2-5 2015 ({\href{https://drive.google.com/open?id=0B8hsJozmBILETlV3VVQ3S21NLTg}{\color{blue} Abstract}})({\href{https://drive.google.com/open?id=0B8hsJozmBILEcGRiT2VzZ0hFUEE}{\color{blue} Here}})

\section{\sc Selected Projects}
{\bf Radioactive-Decay Simulator}\\
This project was done as a part of the paper written for the XXVII IUPAP CCP (2015) in which various Monte-Carlo 
simulations were obtained for the radioactive-decay phenomenon. This project in its initial stages was purely
implemented in C++ and plots were obtained using a seperate spreadsheet software. Later studies suggested that the entire 
process can automated via a numerical-computation tool such as Scilab. It can further be implemented in Python with
the relevant libraries (matplotlib, NumPy)\\

{\bf {\href{https://github.com/sominwadhwa/WebCrawlers}{\color{blue} Web Crawler}}}*\\
Some standard python scripts that use the beautifulSoup library to crawl through various websites related to major
sports leagues and fetch real-time standings of the respective teams. This project was done primarily to fetch the 
league standings for National Basketball Association (NBA) but the functionality can be extended as well.


{\bf {\href{https://github.com/sominwadhwa/UdStudentData}{\color{blue} UdStudentData}}}\\
Rudimentary data analysis of some student data obtained from www.udacity.com. Analysis was solely done in python and
features (analysed \& plotted) variations in Time, Lessons Completed \& Number of Days of student visits in a particular course. Entire analysis is based on three parameters\- Enrollments, Daily Engagements \& Project Submissions.


{\bf Spam-Classifier}*\\
A filter that is able to classify emails as spam or not spam with high accuracy. Entire project is based on {\bf Support Vector Machines}. Full scale implementation of project involves pre-processing of email text and extraction of features from the same. This data is then used to train a SVM with linear kernel to generate the required parameters for classification. Currently working in Matlab.\\
\\ \textit{*Ongoing Projects}
\\ All the projects (including the above-mentioned) are/will be available on GitHub.


\section{\sc Other Achievements}
\begin{list2}

\item {\bf Secretary}( 2015-present)  `Association of Computing Machinery' at MAIT. ({\href{https://mait.acm.org/team.html}{\color{blue} here}})
\item {\bf Interned} at a national NGO `Umeed - A drop of Hope' (NGO Reg: S/792/DIST.SOUTH/201) and jointly participated in Project- Knowledge for All (KFA).
\item {\bf Rotaractor} (2014-2015) Member of `Rotaract Club of Delhi Akash' where I jointly organized
several large scale events like `CanSupport's Walk of Life (8th Feb 2015) - Fight against cancer.', 
`Patrika - A paper recycling drive.'
\item {\bf Volunteer at Techsurge \& Mridang} Annual technical \& cultural fest of MAIT

\end{list2}

\section{\sc Hobbies \& Interests}
Reading about Economics, Basketball, Watching Documentaries, Quora, {\href{https://www.hackerrank.com/sominwadhwa}{\color{blue} HackerRank}}



\section{\sc References }
Available upon request.

\end{resume}
\end{document}




