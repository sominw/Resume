\tolerance=1
\emergencystretch=\maxdimen
\hyphenpenalty=10000
\hbadness=10000

\documentclass[margin,line]{res}
\usepackage[breaklinks=true]{hyperref}
\usepackage{url}
\usepackage{color}
\usepackage{enumitem}
\oddsidemargin -.5in
\evensidemargin -.5in
\textwidth=6.0in
\itemsep=0in
\parsep=0in
\topmargin=-0.4in
\topskip=0in
\textheight=10in
 
\newenvironment{list1}{
  \begin{list}{\ding{113}}{%
      \setlength{\itemsep}{0in}
      \setlength{\parsep}{0in} \setlength{\parskip}{0in}
      \setlength{\topsep}{0in} \setlength{\partopsep}{0in}
      \setlength{\leftmargin}{0.17in}}}{\end{list}}
\newenvironment{list2}{
  \begin{list}{$\bullet$}{%
      \setlength{\itemsep}{0in}
      \setlength{\parsep}{0in} \setlength{\parskip}{0in}
      \setlength{\topsep}{0in} \setlength{\partopsep}{0in}
      \setlength{\leftmargin}{0.2in}}}{\end{list}}


    
\begin{document}

\name{\LARGE Somin Wadhwa} \hfill {\em \today}

\begin{resume}
\section{\sc Contact Information}

\vspace{.025in}
\begin{tabular}{@{}p{3.5in}p{3in}}
{E-mail:}  \href{mailto:sominwadhwa@gmail.com}{sominwadhwa@gmail.com} & {GitHub/Kaggle:} {\href{https://github.com/sominwadhwa}{\underline{/sominwadhwa}}}\\
{Webpage:} {\href{https://sominwadhwa.github.io}{sominwadhwa.github.io}} & {LinkedIN:} {\href{https://www.linkedin.com/in/sominwadhwa/}{\underline{/in/sominwadhwa}}}\\
\end{tabular}

\section{\sc Education}
{\bf Bachelor of Technology in Computer Science \& Engineering} \hfill 2014 -- 2018\\
Maharaja Agrasen Institute of Technology \hfill CPI: {\bf 79.2\%}\\
Guru Gobind Singh Indraprastha University, Delhi, India \hfill GPA (WES): 3.71/4

\section{\sc Recent Experience}
\begin{itemize}[leftmargin=*]
\item {\bf Indraprastha Institute of Information Technology, Delhi (IIIT-D)}\\
\textit{Research Intern} \hfill July, 2018 - Present\\
Complex Systems Lab, Center for Computational Biology \hfill June 2017 - March 2018 \\  
{\bf Advisor}: {\href{https://scholar.google.co.in/citations?user=qyth_0QAAAAJ&hl=en}{\underline{Prof. Ganesh Bagler}}}
\begin{itemize}[leftmargin=*]
\item {\bf Current Work:} My work focuses on creation of a database, \textit{``BitterSweet: A resource to explore and predict taste information in small molecules"}. ({\href{http://cosylab.iiitd.edu.in/bittersweet/}{http://cosylab.iiitd.edu.in/bittersweet}})
\item {\bf Previous Work:} Pursued an internship where I worked on devising new methods to predict phenotypic side effects of drugs using existing data (SIDER4 \& DrugBank). The problem was formulated as an extreme multiclass-multilabel classification problem with severe class imbalance in the datasets. The work conducted was culminated in the form of a research article (published) and the code was documented \& is now open-sourced on Github ({\href{https://github.com/sominwadhwa/drugADR}{\underline{code}}}).
\end{itemize}
\item {\bf All India Council for Technical Education (AICTE), Govt. of India}\\
\textit{Research} \& \textit{Development Intern} \hfill October 2017 - March 2018\\  
{\bf Advisor}: Dr. N.H. Siddalingaswamy (Director, AICTE)
\begin{itemize}[leftmargin=*]
\item {\bf Work:} Lead a team of 5 with a project budget of \$4600 (Rs. 300,000) sponsored by the Ministry of Human Resources Development, Government of India. We curated and analysed graduate employment statistics of several years and developed dynamic dashboards to aid AICTE in granting approvals to higher education institutions. ({\href{https://github.com/TeamExtrapolate/extrapolate}{\underline{code}}})
\end{itemize}
\end{itemize}

\section{\sc Publications}
\begin{list2}
\item Tuwani R, {\bf Wadhwa S}, Bagler G (2018) BitterSweet: Building machine learning models for predicting the bitter and sweet taste of small molecules. bioRxiv 426692 \textit{(preprint)}
\\doi: {\href{https://doi.org/10.1101/426692}{https://doi.org/10.1101/426692}}\\
\item {\bf Wadhwa S}, Gupta A, Dokania S, Kanji R, Bagler G (2018) A hierarchical anatomical classification schema for prediction of phenotypic side effects. PLOS ONE 13(3): e0193959
\\doi: {\href{https://doi.org/10.1371/journal.pone.0193959}{https://doi.org/10.1371/journal.pone.0193959}}
\end{list2}

\section{\sc Selected Projects}
All of my projects (including the following) are available on {\href{https://github.com/sominwadhwa}{github.com/sominwadhwa}}\\
\begin{itemize}[leftmargin=*]
\item {\bf {\href{https://github.com/sominwadhwa/vqamd_floyd}{Visual Question Answering through Modal Dialogue}:}}\\
A semester long B.Tech project based on the application of \textit{\href{https://arxiv.org/pdf/1505.01121.pdf}{\underline{Malinowski et al.}}} (CNN + LSTM based models) on v2 of the {\href{http://visualqa.org/}{\underline{VQA}}} dataset released in April 2017. Achieved a total accuracy of 54.88\% with the best model. I have also documented and made the entire process reproducible with a single click in the form of a \href{https://blog.floydhub.com/asking-questions-to-images-with-deep-learning/}{\underline{featured blog post}}. ({\href{https://github.com/sominwadhwa/vqamd_floyd}{\underline{code}}})
\item {\bf {\href{https://github.com/sominwadhwa/Kaggle}{Kaggle Repository}:}}\\
An ongoing (2+ years) collection of kernels (implemented using IPython notebooks) designed using datasets obtained from Kaggle for practise and competitions. It demonstrates the implementation of several algorithms and data visualization techniques. ({\href{https://github.com/sominwadhwa/Kaggle}{\underline{github-repo}}}, {\href{https://www.kaggle.com/sominwadhwa}{\underline{kaggle profile}}})
\item {\bf {\href{https://github.com/sominwadhwa/TheTwitterPolice}{TheTwitterPolice}:}}\\
A basic analysis \& visualization of the Indian law enforcement activity on Twitter. Collected data for different cities (BeautifulSoup \& Selenium), stored them in a database (MongoDB), analysed (sentiment analysis, basic statistics etc) \& displayed the results graphically through a flask web-application. ({\href{https://github.com/sominwadhwa/TheTwitterPolice}{\underline{code}}})
\item {\bf {\href{https://github.com/sominwadhwa/Image-Apportionor}{Image Apportionor}:}}\\
A simple clustering based image segmentation project. Implemented the k-means clustering algorithm, from scratch, in Python as part of the academic summer activity requirement in my sophomore year. ({\href{https://github.com/sominwadhwa/TheTwitterPolice}{\underline{code}}})
\end{itemize}

\iffalse
\section{\sc Technical Skills}
{\bf Strongest Areas}: Machine Learning (Classification, Regression, Feature Engineering), Algorithms, Statistical Data Analysis\\
{\bf Languages/Tools/Software}: Python (scikit-learn, Keras, NumPy, Pandas \& others), Java, SQL, MongoDB, \LaTeX, MS Excel
\fi

\section{\sc Skills}
\begin{list2}
\item {\bf Languages}: Python (proficient), Java (familiar), bash scripting, SQL, \LaTeX. 
\item {\bf Frameworks \& Libraries}: Tensorflow, keras, PyTorch, scikit-learn, SpaCy, NLTK, Matplotlib, Plotly, MongoDB, Flask.
\item {\bf Relevant Classes Taken}: Data Structures, Algorithms Design \& Analysis, Probability \& Curve Fitting (Applied Mathematics-IV), Advanced DBMS, Machine Learning, Soft Computing.
\item {\bf MOOCs}: Deep Learning Specialization by Andrew Ng (5 courses, {\href{https://www.coursera.org/account/accomplishments/specialization/WK4DU58W9DV5}{\underline{certificate}}}), Algorithms Design \& Analysis by Tim Roughgarden ({\href{https://www.coursera.org/account/accomplishments/verify/S8RN7R4ZZQSR}{\underline{certificate}}}), Machine Learning by Andrew Ng ({\href{https://www.coursera.org/account/accomplishments/verify/37T4VG9ZN7EY}{\underline{certificate}}}).
\end{list2}

\section{\sc Achievements \& \\ Other Activities}
\begin{list2}
\item {\bf Outstanding Achievement Award}: Conferred by the Computer Science \& Engineering department at MAIT, Delhi for my contributions to the institute.
\item {\bf Best B.Tech Project}: My senior year project on Visual Question Answering was selected among the top four out of the 60+ projects reviewed by the department.
\item {\bf Kaggle Kernels Expert}: Elevated on Kaggle with a worldwide rank 135 (out of 78,000+ competitors).
\item {\bf Kaggle Discussions Expert}: Elevated on Kaggle with a worldwide rank 122 (out of 70,000+ competitors).
\item {\bf Smart India Hackathon 2017, MHRD, Govt. of India}: Lead a team of 6-members advised by {\href{https://www.linkedin.com/in/sambuddharoy/}{Dr. Sambuddha Roy (Principal Data Scientist at Microsoft, Seattle)}} and {\href{https://www.linkedin.com/feed/update/urn:li:activity:6255398180318470144}{\bf {\underline{won first prize}}}} with a total reward of \$3000 (Rs. 200,000) awarded by Government of India and MAIT, Delhi. As part of the winning team, we were also provided with an oppotunity to take our prototype forward and this resulted in a 6-month long internship with the All India Council for Technical Education (AICTE).
\item {\bf Secretary, Association of Computing Machinery}: Co-founded \& served in the capacity of Secretary of 80+ strong team of ACM-MAIT Student Chapter during 2015-2016.
\item {\bf Community Volunteer} at a national NGO `Umeed - A drop of hope' (NGO Reg.: S/792/DIST.SOUTH/201) and participated in the project - Knowledge for all (KFA).
\item {\bf Rotaractor}: Member of `Rotaract Club of Delhi Akash' during 2014-2015 where our team jointly organized several large scale events like `Walk of Life', `Fight Against Cancer', `Patrika - A paper recycling drive'. 
\item {\bf Blogging}: Maintaining an active blog at \textit{{\href{https://sominwadhwa.github.io/blog/}{sominwadhwa.github.io/blog}}} to document some of my experiences \& selected projects (for reproducibility).
\end{list2}

\section{\sc References}
\begin{list2}
\item Dr. Namita Gupta, Head of Department, Computer Science \& Engineering, MAIT, Delhi\\ E-mail: namita@mait.ac.in
\item Dr. Ganesh Bagler, Complex Systems Lab, Center for Computational Biology, IIIT-Delhi\\ E-mail: bagler@iiitd.ac.in
\end{list2}

\end{resume}
\end{document}
