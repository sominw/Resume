\documentclass[margin,line]{res}
\usepackage{hyperref}
\usepackage{url}
\usepackage{color}
\usepackage{enumitem}
\oddsidemargin -.5in
\evensidemargin -.5in
\textwidth=6.0in
\itemsep=0in
\parsep=0in
\topmargin=-0.4in
\topskip=0in
\textheight=10in
 
\newenvironment{list1}{
  \begin{list}{\ding{113}}{%
      \setlength{\itemsep}{0in}
      \setlength{\parsep}{0in} \setlength{\parskip}{0in}
      \setlength{\topsep}{0in} \setlength{\partopsep}{0in}
      \setlength{\leftmargin}{0.17in}}}{\end{list}}
\newenvironment{list2}{
  \begin{list}{$\bullet$}{%
      \setlength{\itemsep}{0in}
      \setlength{\parsep}{0in} \setlength{\parskip}{0in}
      \setlength{\topsep}{0in} \setlength{\partopsep}{0in}
      \setlength{\leftmargin}{0.2in}}}{\end{list}}


    
\begin{document}

\name{\LARGE Somin Wadhwa} \hfill {\em \today}

\begin{resume}
\section{\sc Contact Information}

\vspace{.025in}
\begin{tabular}{@{}p{3.5in}p{3in}}
{E-mail:}  \href{mailto:sominwadhwa@gmail.com}{sominwadhwa@gmail.com} & {GitHub/Kaggle:} {\href{https://github.com/sominwadhwa}{\underline{/sominwadhwa}}}\\
{Homepage:} {\href{https://sominwadhwa.github.io}{sominwadhwa.github.io}} & {LinkedIN:} {\href{https://www.linkedin.com/in/sominwadhwa/}{\underline{/in/sominwadhwa}}}\\
\end{tabular}

\section{\sc Education}
{\bf Bachelor of Technology in Computer Science \& Engineering} \hfill  2014 -- 2018\\
Maharaja Agrasen Institute of Technology \hfill CPI: {\bf 79.2\%}\\
Guru Gobind Singh Indraprastha University, Delhi, India

\section{\sc Recent Experience}
\begin{itemize}[leftmargin=*]
\item {\bf Indraprastha Institute of Information Technology, Delhi (IIIT-D)}\\
\textit{Research Intern} \hfill July, 2018 - Present\\
Complex Systems Lab, Center for Computational Biology \hfill June 2017 - March 2018 \\  
{\bf Advisor}: {\href{https://scholar.google.co.in/citations?user=qyth_0QAAAAJ&hl=en}{\underline{Dr. Ganesh Bagler}}}
\begin{itemize}[leftmargin=*]
\item {\bf Current Work:} My current work focuses on creation of \textit{``BitterSweet: A resource to explore and predict taste information in small molecules"}. ({\href{http://cosylab.iiitd.edu.in/bittersweet/}{http://cosylab.iiitd.edu.in/bittersweet}})
\item {\bf Previous Work:} During my undergraduate studies, we worked on devising new methods to predict phenotypic side effects of drugs using existing data (SIDER4). The work was culminated in the form of a research article and the code was documented \& open-sourced on Github ({\href{https://github.com/sominwadhwa/drugADR}{\underline{link}}}).
\end{itemize}
\item {\bf All India Council for Technical Education (AICTE), Govt. of India}\\
\textit{Research} \& \textit{Development Intern} \hfill October 2017 - March 2018\\  
{\bf Advisor}: Dr. N.H. Siddalingaswamy (Director, AICTE)
\begin{itemize}[leftmargin=*]
\item {\bf Work:} Lead a team of 5 and with a project budget of \$4600. We analysed AICTE's employment statistics dataset and developed dynamic analytic dashboards to aid AICTE in granting approvals to higher education institutions.
\end{itemize}
\end{itemize}

\section{\sc Publications}
{\bf Wadhwa S}, Gupta A, Dokania S, Kanji R, Bagler G (2018) A hierarchical anatomical classification schema for prediction of phenotypic side effects. PLOS ONE 13(3): e0193959. 
\\{\href{https://doi.org/10.1371/journal.pone.0193959}{\underline{https://doi.org/10.1371/journal.pone.0193959}}}

\section{\sc Selected Projects}
All of my projects (including the following ones) are be available on {\href{https://github.com/sominwadhwa}{\underline{github.com/sominwadhwa}}}
\begin{itemize}[leftmargin=*]
\item {\bf {\href{https://github.com/sominwadhwa/vqamd_floyd}{Visual Question Answering through Modal Dialogue:}}}
A two semester long B.Tech project based on the application of \textit{\href{https://arxiv.org/pdf/1505.01121.pdf}{Malinowski et al.}} on v2 of the {\href{http://visualqa.org/}{\underline{VQA}}} dataset. Documented and made the entire process reproducible in the form of a \href{https://blog.floydhub.com/asking-questions-to-images-with-deep-learning/}{\underline{featured blog post}}. ({\href{https://github.com/sominwadhwa/vqamd_floyd}{\underline{code}}})
\item {\bf {\href{https://github.com/sominwadhwa/Kaggle}{Kaggle Repository}:}}\\
A collection of kernels (written in IPython Notebooks \& scripts) designed from datasets obtained from Kaggle for practise as well as competitions. These include implementations of typical Machine Learning algorithms on a range of datasets.
\item {\bf {\href{https://github.com/sominwadhwa/TheTwitterPolice}{\color{blue} TheTwitterPolice}}}\\
Analysis of law enforcement activity on Twitter in India. Collected data from five different police social handles (BeautifulSoup \& Selenium), stored them in a database (MongoDB), analysed (sentiment-analysis, time-series etc) \& displayed the results graphically in the form of a web-app (flask application deployed on heroku).
\end{itemize}

\iffalse
\section{\sc Technical Skills}
{\bf Strongest Areas}: Machine Learning (Classification, Regression, Feature Engineering), Algorithms, Statistical Data Analysis\\
{\bf Languages/Tools/Software}: Python (scikit-learn, Keras, NumPy, Pandas \& others), Java, SQL, MongoDB, \LaTeX, MS Excel
\fi

\section{\sc Other Activities}
\begin{list2}
\item {\href{https://www.linkedin.com/feed/update/urn:li:activity:6255398180318470144}{\color{blue} Won Smart India Hackathon}} (April 2017) I was the Team Lead of a six-member team under the mentorship of {\href{https://www.linkedin.com/in/sambuddharoy/}{\color{blue} Dr. Sambuddha Roy}} over a period of three months to build a decision support system using Machine Learning to improvise AICTE's handbook approval system for technical institutions in India for {\href{https://www.linkedin.com/pulse/smart-india-hackathon-2017-anirban-sarker}{\color{blue} SIH}} -- 7200+ teams pan India competed in a 36-hour Hackathon organised by Government of India. As a part of the winning team for AICTE, I'm associated with All India Council for Technical Education (Ministry of Human Resources \& Development, Government of India) in a fully funded project ({\bf Budget: 2.93L}) for taking our prototype forward over the period of 6-8 months beginning October 2017. 
\item {\bf Secretary}(2015-2016)  `Association of Computing Machinery (ACM)- Student Chapter' at M.A.I.T
\item {\bf {\href{https://drive.google.com/open?id=0B8hsJozmBILEcGRiT2VzZ0hFUEE}{\color{blue} Presentation}}} Gave an oral talk on, ``Study of Random Numbers \& their applications in computational physics using Monte-Carlo method" at the 27\textsuperscript{th} IUPAP Conference on Computational Physics, {\bf IIT Guwahati} on 2-5 December, 2015.
\item {\bf Interned} at a national NGO `Umeed - A drop of Hope' (NGO Reg: S/792/DIST.SOUTH/201) and participated in Project- Knowledge for All (KFA).
\item {\bf Rotaractor} (2014-2015) Member of `Rotaract Club of Delhi Akash' where our team jointly organized several large scale events like `CanSupport's Walk of Life (8th Feb 2015) - Fight against cancer.', 
`Patrika - A paper recycling drive.'
\end{list2}
\end{resume}
\end{document}
